%revisar
\subsection{Canelones}
\title{Canelones}
\leftbgpic{pasta/canelones}
\dline
\begin{info}
 Raciones: 4 personas \\
 Tiempo: 40 minutos
\end{info}
\recipesection{Ingredientes}
\dline
\begin{ingredients}
 \item{Pasta de canelones (16 planchas)}
 \item{300g de carne picada}
 \item{1 lata de pate}
 \item{\nicefrac{1}{2} cebolla}
 \item{2 ajos}
 \item{4 cucharadas de tomate frito}
 \item{60g de harina}
 \item{60g de mantequillas}
 \item{600ml de leche}
 \item{1 pizca de sal}
 \item{1 pizca de nuez moscada}
 \item{Queso rallado para gratinar}
\end{ingredients}

\recipesection{Realización}
\dline
\step{1.}{
Pocha la cebolla con los ajos, incorpora la carne,
el pate y el tomate frito y rehogarlo.
}
\step{2.}{
 Pasarlo por la batidora y reservar
}
\step{3.}{
 Cocer la pasta en abundante
 agua con sal y un chorrito de aceite para que no se pegue el
 tiempo que indique el fabricante, enfriar en agua fría
 y escurrir.
 }
 \step{4.}{
 Extender las placas sobre la mesa, rellenarlas
 con la farsa de la batidora y colocar en una bandeja untada
 con mantequilla y tomate frito.
 }
 \step{5.}{
 Hacer una bechamel con la
 mantequilla, la harina, la leche y la nuez moscada. Cubrir los canelones y
 espolvorear el queso rallado. Hornear durante 15-20 minutos a 200º Grados
}
