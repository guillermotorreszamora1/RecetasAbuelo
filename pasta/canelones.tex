%revisar
\subsection{Canelones}
\title{Canelones}
\leftbgpic{pasta/canelones}
\begin{info}
 Raciones: 4 personas \\
 Tiempo: 40 minutos
\end{info}
\recipesection{Ingredientes}
\begin{ingredients}
 \item{Pasta de canelones (16 planchas)}
 \item{300g de carne picada}
 \item{1 lata de pate}
 \item{\nicefrac{1}{2} cebolla}
 \item{2 ajos}
 \item{4 cucharadas de tomate frito}
 \item{60g de harina}
 \item{60g de mantequillas}
 \item{600ml de leche}
 \item{1 pizca de sal}
 \item{1 pizca de nuez moscada}
 \item{Queso rallado para gratinar}
\end{ingredients}
\recipesection{Realización}
\step{1.}{
Pocha la cebolla con los ajos incorporar la carne,
el pate y el tomate , rehogarlo todo bien y pasarlo
 por la batidora y apartar, cocer la pasta en abundante
 agua y un chorrito de aceite para que no se pegue el
 tiempo que indique el fabricante , enfriar en agua fría
 y escurrir .Extender las placas sobre la mesa , rellenarlas
 con la farsa de la batidora colocar en una bandeja , untada
 con mantequilla y tomate frito . Hacer una bechamel con la
 mantequilla, la harina, la leche y la nuez moscada, cubrir los canelones y
 espolvorear el queso rallado y hornear durante 15-20minutos
 a 200º Grados
}
