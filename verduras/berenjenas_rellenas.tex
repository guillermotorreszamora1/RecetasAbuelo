\subsection{Berenjenas Rellenas}
\title{Berenjenas Rellenas}
\leftbgpic{verduras/calabacin_relleno1}
\dline
\begin{info}
 Raciones: 6 personas \\
 Tiempo de preparación: ??
\end{info}
\recipesection{Ingredientes}
\dline
\begin{ingredients}
\item{5-6 berenjenas}
\item{\nicefrac{1}{2} kg de carne picada mixta}
\item{1 cebolla}
\item{4 o 5 cucharadas de Salsa de tomate frito}
\item{Salsa bechamel (Ver receta)}
\item{Queso rallado}
\item{Sal}
\item{Pimienta}
\end{ingredients}

\recipesection{Realización}
\dline
\step{1.}{
Lavar las berenjenas, partirlas por la mitad y hacer cortes en horizontal y vertical en la carne de la berenjena. Echarles un chorito de aceite y meterlas en el 
horno a 180º durante 30-40 minutos hasta que la carne este tierna (Este paso también se puede hacer en el microondas reduciendo el tiempo. Se ponen en un bol y se tapan con
papel film).
}
\step{2.}{
Una vez asadas las berenjenas dejar enfriar un poco y vaciar la carne de la berenjena con un cuchara y picarla bien.
}
\step{3.}{
En una sartén, pochar la cebolla y añadir la carne de las berenjenas y la carne picada previamente salpimentada. Dejar que se cocine, removiendo de vez en cuando y añadir la salsa de tomate.
}
\step{4.}{
Rellenar las berenjenas, cubrir de bechamel, espolvorear con el queso rallado y dorar en el horno con la opción grill unos 15 minutos a 200º.
}

\recipesection{Notas}
El relleno se puede sustituir por atún en aceite o solo con verduras(Champiñones, calabacín).
Si quieres aligerar la receta en vez de bechamel puedes cubrirlo con pan rallado.
Este receta se puede hacer también con calabacín.
