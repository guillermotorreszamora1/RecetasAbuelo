
\subsection{Albóndigas caseras}
\title{Albóndigas caseras}
\leftbgpic{Vacuno/albondigas2.jpg}
\dline
\begin{info}
Raciones: 6 personas \\
Tiempo de preparación: 1 hora
\end{info}
\recipesection{Ingredientes}
\dline
\begin{ingredients}
\item{\nicefrac{3}{4} kg de carne de vacuno}
\item{4 ajos}
\item{sal}
\item{Pimienta}
\item{Perejil}
\item{1 huevo}
\item{4 rebanas de pan bimbo sin corteza}
\item{\nicefrac{1}{2} cebolla}
\item{\nicefrac{1}{2} vaso de leche}
\item{Harina}
\item{Aceite}
\item{1 vasito de vino blanco}
\item{2 Patatas gordas}
\item{1 botecito de guisantes}
\end{ingredients}

\recipesection{Realización}
\dline
%\begin{comment}
\step{1.}{
Adobar la carne: en el vaso de la batidora, añadir dos ajos picados, un huevo, perejil, dos rebandas de pan de molde y un vaso de leche.
Pasar por la turmix y agregarlo a la carne previamente salpimentada y amasarlo bien.
}
\dline
\step{2.}{
Partir las zanahorias en cuadraditos pequeños y ponerlas a hervir en 1 litro de agua con una pastilla de caldo de pollo.
}
\dline
\step{3.}{
Cortar las patatas en cuadraditos y freirlas
}
\dline
\step{4.}{
Formar las albóndigas, enharinarlas y freirlas en el aceite de las patatas y colorcarlas en una cazuela.
}
\dline
\step{5.}{
Picar la cebolla y pocharla en el aceite donde hemos frito las albóndigas(retirando el aceite sobrante). Mientras machaca los dos ajos
con perejil y un vaso de vino blanco. Cuando la cebolla este pochada, añadir una cucharada de harina y el majado. Dejar evaporar un
par de minutos y añadir las zanahorias con el caldo. Verterlo en la cazuela de las albóndigas. Agregar los guisantes y las patatas y dejar cocer 10 minutos
}
%\end{comment}
\begin{comment}
\step{1.}{
	Adoba la carne con 2 ajos,  un huevo, sal y pimienta, perejil,  
y la leche con el pan remojado  pasando los ingredientes por 
la túrmix, agrégalo a la carne y amásalo bien.
}

\step{2.}{
 Parte las zanahorias en cuadritos pequeños y ponlas a hervir (opcional puedes 
añadirle 1 pastilla de caldo ) corta las patatas en cuadritos dóralas 
y apártalas , forma las albóndigas, pásalas por harina y fríelas 
ligeramente , sácalas y ponlas en una cazuela , pica la cebolla fina ,
retira aceite de la sartén y póchalas suavemente , machaca los 2 ajos 
que quedan , con un poco de perejil y vino blanco , cuando la cebolla 
este pochada agrega 1 cucharada de harina , incorpora el majado (los ajos, 
el perejil y el vino blanco) a la cebolla , agrega las zanahoria con el 
caldo que tengas a la sartén de la cebolla y viértelo en la cazuela de
 las albóndigas ,si hace falta , cubre con agua, agrega la latita de guisantes 
 y cuécelo durante 1/2 hora a fuego lento , incorporando las patatas cuando falte 
 1/4 de hora y servir
}
\end{comment}
