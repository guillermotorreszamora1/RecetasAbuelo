%revisar
\subsection{Pimientos rellenos}
\title{Pimientos relle-\\nos}
\dline
\leftbgpic{verduras/pimientosrellenos2}
\begin{info}
Raciones: 4 personas \\
Tiempo de elaboración: 1h \\ 15 minutos
\end{info}
\recipesection{Ingredientes}
\dline
\begin{ingredients}
 \item{12 pimientos de piquillo}
 \item{\nicefrac{1}{2} cebolla}
 \item{3 tomates pelados}
 \item{1 pimiento verde pequeño}
 \item{2 ajos}
 \item{400g de bacalao desalado o gambas}
 \item{60g de mantequilla}
 \item{60g de harina}
 \item{200g de nata}
 \item{600ml de leche}
 \item{Sal}
 \item{Nuez moscada para aromatizar}
\end{ingredients}

\recipesection{Realización}
\dline
\step{1.}{
Hacer una bechamel con 60g de mantiquilla, 60g de harina y medio litro de leche(Ver receta).
}
\step{2.}{
Saltear el bacalao, las gambas o el relleno que se le quiera echar(Puerro, setas, ...).
}
\step{3.}{
Añadir el relleno a la bechamel y remover.
}
\step{4.}{
Para la salsa, pochar la cebolla, el ajo y el pimiento verde.
Añadirle dos pimientos del piquillo partidos en trozos y un poco de oregano
Una vez pochado añadir una copa de brandy. Reducir unos minutos y añadirle 5 cucharadas de tomate frito y la nata.
}
\step{5.} {
Batir la salsa.
}
\dline
\step{6.}{
Rellenar los pimientos con la bechamel y cubrir con la salsa.
}
