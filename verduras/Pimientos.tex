%revisar
\subsection{Pimientos rellenos}
\title{Pimientos rellenos}
\leftbgpic{verduras/pimientos}
\begin{info}
Raciones: ?? personas
\end{info}
\recipesection{Ingredientes}
\begin{ingredients}
 \item{12 pimientos de piquillo}
 \item{\nicefrac{1}{2} cebolla}
 \item{3 tomates pelados}
 \item{1 pimiento verde pequeño}
 \item{2 ajos}
 \item{400g de bacalao desalado}
 \item{60g de mantequilla}
 \item{60g de harina}
 \item{200g de nata}
 \item{600ml de leche}
 \item{Sal}
 \item{Nuez moscada para aromatizar}
\end{ingredients}
\recipesection{Realización}
\dline
\step{1.}{
Pocha la cebolla, conjuntamente
 con el pimiento verde y los ajos,
 cuando este blanda incorpora los
 tomates pelados y troceados. Sofríe
 a fuego lento hasta que este tierno.
 Agrega dos pimientos de piquillo en
 trozos y pasa el conjunto por la batidora, añadir
 la nata, batir y reservar para salsear
 los pimientos.
}
