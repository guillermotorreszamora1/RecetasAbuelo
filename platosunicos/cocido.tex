\subsection{Cocido Madrileño}
\title{Cocido Madrileño}
\dline
\leftbgpic{platosunicos/cocido}
\begin{info}
 Raciones: 10 personas \\
 Tiempo de preparación: 4 horas
\end{info}
\recipesection{Ingredientes}
\dline
\begin{ingredients}
 \item{1 kg de jarete o morcillo}
 \item{2 chorizos}
 \item{300g de panceta salada}
 \item{1 hueso de cerdo salado}
 \item{\nicefrac{1}{4} kg de gallina}
 \item{1 punta de jamón}
 \item{700g de garbanzos}
 \item{1 puerro}
 \item{2 zanahorias}
 \item{1 patata}
 \item{1 trozo de repollo}
 \item{1 ajo}
 \item{1 cucharadita de pimentón}
 \item{fideos}
 \item{1 ramita de hierbabuena}
\end{ingredients}

\recipesection{Realización}
\dline
\step{1.}{
Pon la víspera a remojar los garbanzos.
}
\dline
\step{2.}{
En una cazuela grande pon todos
 los ingredientes menos el repollo y los garbanzos.\\
 Ponlo en abundante agua a cocer
 lentamente añadiendo si hace falta agua siempre caliente a la cocción.
 Cuando empieze a hervir añadir los garbanzos en una malla.
 }
 \dline
 \step{3.}{
 Mientras cocer el repollo y una vez cocido escurrir y rehogarlo con ajo y un poquito de pimentón.
 }
 \dline
\step{4.}{
Dejar cocer aproximadamente 3 horas y probar los garbanzos.
 Cuando este cocido apartar las verduras y los garbanzos por un lado y las carnes por otro.
 Colar el caldo y hacer sopa añadiendole fideos, se le puede añadir una hoja de hierbabuena.
}
\dline
\step{5.}{Servir primero la sopa, luego las
 verduras y los garbanzos y después la carne.
}
