\subsection{Redondo de Ternera}
\title{Redondo de Ternera}
\dline
\leftbgpic{Vacuno/Redondo.jpg}
\begin{info}
Raciones: 6 personas \\
Tiempo de preparación: 1 \nicefrac{1}{4} hora
% empezado a las 11:19
\end{info}
\recipesection{Ingredientes}
\dline
\begin{ingredients}
\item{1 rendondo de ternera con malla}
\item{5 zanahorias}
\item{3 puerros}
\item{1 cebolla}
\item{1 patata}
\item{2-3 cucharadas de harina}
\item{1 ajo}
\item{perejil}
\item{1 vaso de vino blanco}
\item{Sal y pimienta}
\item{1 hoja de laruel}
\item{1 pastilla de caldo de pollo(opcional)}
\end{ingredients}

\recipesection{Realizacion}
\dline
\step{1.}{
Salpimentar el redondo y después enharinarlo.
}
\newline
\step{2.}{
En una olla con un poco de aceite, marcar el redondo. Retirar y reservar en una fuente.
En la misma olla donde se ha sellado el redondo, pochar la cebolla, el puerro y la zanahoria.
Mientras se pocha, añadir un poco de sal.
}
\newline
\step{3.}{
Mientras se está pochando la verdura, en un mortero añadir un ajo y perejil, machacar y añadir el vaso de vino blanco.
Verter esto sobre el redondo. Dejar que se evapore un poco y cubrir de agua. Añadir dos granos de pimienta, la patata pelada
y una hoja de laurel. Rectificar el punto de sal o añadir una pastilla de caldo de pollo, cerrar la olla y 40 minutos en la posición 2.
}
\newline
\step{4.}{
Cuando haya terminado la cocción, sacar el redondo. Pasar la salsa. Dejar el redondo enfriar y cortarlo en lonchar finas.
Cubrir con la salsa.
*En caso de que la salsa quede un poco liquida, espesar con maicena (En un vaso de agua fria añadir una cucharada de maicena y
remover hasta disolverla, añadirsela a la salsa y dejar cocer unos minutos).
}


