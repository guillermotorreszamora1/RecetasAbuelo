\subsection{Torrijas}
\title{Torrijas}
\leftbgpic{postres/torrijas4.jpg}
\dline
\begin{info}
 Raciones: 5 personas \\
 Tiempo de preparación: 1 hora
%12:07
\end{info}
\recipesection{Ingredientes}
\begin{ingredients}
\item{1 litro de leche}
\item{12-14 cucharadas de azúcar}
\item{Canela en rama}
\item{Canela molida}
\item{1 barra de pan de torrijas}
\item{2-3 huevos}
\end{ingredients}
\recipesection{Realización}
\dline
\step{1.}{
Poner a calentar el litro de leche con 6 cucharadas de azúcar y una ramita de canela.
Dejar cocer un par de minutos. Partir el pan en rebanadas al gusto de tamaño y ponerlo en un recipiente.
Verter la leche sobre las rebanadas. Dejar que se empapen y darles la vuelta.
}
\dline
\step{2.}{
Batir un par de huevos y pasar las rebanas de pan por el huevo e ir friendo por los dos lados en aceite bien caliente y abundante.
Al sacarlas ponerlas sobre un papel para que pierdan un poco de grasa y mientras siguan calientes rebozarlas en una mezcla que abremos
hecho previamente de azúcar y canela molida.
}
\dline
\dline
\recipesection{Notas}
Si quieres puedes cubrirlas con un poco de leche y azúcar para que esten más jugosas(Se infusiona la leche con el azúcar y una ramita de canela.
