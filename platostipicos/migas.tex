\subsection{Migas de la abuela Jacinta}
\title{Migas de la abuela Jacinta}
\dline
\leftbgpic{platostipicos/migas_4}
\begin{info}
 Raciones: 5-6 personas \\
 Tiempo de preparación: 45 minutos
 
\end{info}
\recipesection{Ingredientes}
\dline
\begin{ingredients}
\item{Una hogaza de pan}
\item{6 dientes de ajo}
\item{Aceite}
\item{Sal}
\end{ingredients}

\recipesection{Realización}
\dline
\step{1.}{
La noche anterior, picar el pan en cuadraditos pequeños, dejarlo reposar en una cacerola y añadir agua para humidificarlo(2 o 3 vasos e ir viendo el tacto).
Hay que removerlo con las manos para que el agua se distribuya bien y tapar con un trapo de cocina.
}
\dline
\step{2.}{
 En una cacerola amplia, añadir aceite que cubra bien la base de la cacerola y añadir un poco más.
}
\dline
\step{3.}{
Picar los ajos en laminas y dorarlos. Añadir el pan, echar sal y empezar a remover. Hay que remover constantemente con la intención de ir 
también partiendo el pan hasta que queden tostadas y sueltas durante aproximadamente 30-40 minutos.
}
\recipesection{Notas}
En la casa de mi abuela se acompañaban con cafe con leche y se tomaban en el desayuno.
